\documentclass[letterpaper,11pt]{article}
\usepackage[spanish]{babel}
\usepackage[utf8]{inputenc}

\usepackage{rotating}
\usepackage{multirow}

\usepackage{lmodern}
\usepackage[T1]{fontenc}
\usepackage{textcomp}

\usepackage[
pdfauthor={Carlos Eduardo Caballero Burgoa},%
pdftitle={Perfil de proyecto},%
colorlinks,%
citecolor=black,%
filecolor=black,%
linkcolor=black,%
urlcolor=black
pdftex]{hyperref}

\title{Perfil de proyecto}
\author{Carlos Eduardo Caballero Burgoa}

\begin{document}
\maketitle
\section{Introducción}
En este documento se plantea el desarrollo de una aplicación web y su versión
móvil que permita acelerar y mejorar el proceso de selección de horarios de
grupos y materias en la facultad de ciencias y tecnología.

\section{Antecedentes}
Cada semestre en la facultad de ciencias y tecnología, se pasa por un proceso
clásico: compra de matricula, publicación de horarios, e inscripción en el
websiss.

En semestres bajos e intermedios, el proceso de seleccionar las materias que uno
va a tomar para el semestre, posee un gran esfuerzo de análisis, para que todas
estas materias no colisionen, y que además (si se puede), posean características
que le convengan al estudiante, según sus propios criterios y disponibilidades
de tiempo.

En este proceso, se toman en cuenta muchas cosas entre otras:

\begin{itemize}
\item Minimizar las colisiones entre horarios.
\item Preferencia por un grupo en especifico.
\item Reducción de los puentes entre clases.
\item Restringir según los tiempos de disponibilidad que se posee.
\end{itemize}

Todo esto concluye a realizar de este proceso, un trabajo moroso y hasta
cierto punto agobiante, particularmente cuando se tienen muchas posibilidades.

\section{Definición del problema}
Se ha vivido y observado que el proceso de inscripción en la facultad de
ciencias y tecnología, realizado dos veces por año presenta algunos
inconvenientes particularmente para aquellos estudiantes de nuevo ingreso o
aquellos que toman una gran cantidad de materias, entre los problemas se
destacan los siguientes:

Los horarios se publican con poca antelación y con un procedimiento poco
flexible, lo que conlleva un proceso tedioso y una inversión considerable de
tiempo.

Los estudiantes representan un conjunto muy variado de restricciones y
condiciones en la toma de decisión acerca de que grupos y materias requieren
tomar, haciendo del proceso aun mas laborioso.

Algunos estudiantes requieren conformar grupos en sus materias lo que
requiere coordinación entre múltiples estudiantes y un monitoreo constante de la
disponibilidad.

Posteriormente al proceso de inscripción también se necesita revisar los
horarios seleccionados mientras el estudiante se acostumbre a este.

Por lo mencionado se define el problema como:

\emph{“El proceso de inscripción estudiantil de la facultad de ciencias y 
tecnología se realiza de forma tediosa, manual, y con tendencia a la
ineficiencia.”}

\section{Objetivos}

\subsection{Objetivo General}
Facilitar el proceso de elección de horarios en tiempo de inscripción a los
estudiantes de la facultad de ciencias y tecnología, a través del desarrollo de
una aplicación web y su versión móvil para la organización de las materias.

\subsection{Objetivos Específicos}
\begin{itemize}
\item Realizar la notificación de los eventos del proceso de inscripción y
    mejorar la accesibilidad a la información.
\item Crear un sistema de reglas que permita validar las restricciones que el
    estudiante establezca.
\item Facilitar de espacios de comunicación y coordinación entre los estudiantes
    durante el proceso de inscripción.
\item Persistir la información del horario del estudiante para que esta pueda ser
    utilizado a modo de consulta.
\end{itemize}

\section{Ingeniería de proyecto}
Véase el cuadro \ref{ingenieriadeproyecto} en la página
\pageref{ingenieriadeproyecto}.

\begin{sidewaystable}
\centering
\small
\begin{tabular}{|l|l|l|p{6.5cm}|l|}
\hline
Objetivo General & Causa & Objetivos Específicos & Actividades & Resultados \\
\hline
\multirow{12}{2.6cm}{Facilitar el proceso de elección de horarios en tiempo de
inscripción a los estudiantes de la facultad de ciencias y tecnología, a través
del desarrollo de una aplicación web y su versión móvil para la organización de
las materias.} &
\multirow{3}{3cm}{Los horarios se publican con poca antelación y con un
procedimiento poco flexible.} &
\multirow{3}{3.5cm}{Realizar la notificación de los eventos del proceso de
inscripción y mejorar la accesibilidad a la información.} &
Diseñar e implementar un sistema de notificación en tiempo real sobre eventos
del proceso de inscripción. &
\multirow{3}{2.5cm}{Modulo de notificación de eventos.} \\
\cline{4-4}
& & & Diseñar e implementar un extractor de información de los horarios durante
el proceso de inscripción. & \\
\cline{4-4}
& & & Diseñar e implementar el API de comunicación del sistema desarrollado.& \\
\cline{2-5}
& \multirow{3}{3cm}{Se presentan múltiples restricciones y condiciones en la
toma de decisiones.} &
\multirow{3}{3.5cm}{Crear un sistema de reglas que permita validar las
restricciones que el estudiante establezca.} &
Analizar los métodos posibles para el diseño del sistema de control de
restricciones. &
\multirow{3}{2.5cm}{Modulo para el control de restricciones.} \\
\cline{4-4}
& & & Diseñar e implementar un sistema para el registro de restricciones
horarias y sus ponderaciones de valor. & \\
\cline{4-4}
& & & Diseñar e implementar la validación de restricciones sobre los horarios
disponibles. & \\
\cline{2-5}
& \multirow{3}{3cm}{Se requiere coordinación y un monitoreo constante entre
estudiantes y la disponibilidad de los grupos.} &
\multirow{3}{3.5cm}{Facilitar de espacios de comunicación y coordinación entre
los estudiantes durante el proceso de inscripción.} &
Crear espacios virtuales para la discusión entre usuarios durante el proceso de
inscripción. &
\multirow{3}{2.5cm}{Modulo de conectividad entre usuarios del sistema.} \\
\cline{4-4}
& & & Diseñar e implementar métodos que mejoren los canales de comunicación. & \\
\cline{4-4}
& & & Crear un API funcional para el intercambio de información. & \\
\cline{2-5}
& \multirow{3}{3cm}{Se necesita tener disponible los horarios finalmente
seleccionados para su consulta durante el periodo de clases.} &
\multirow{3}{3.5cm}{Persistir la información del horario del estudiante para que
esta pueda ser utilizado a modo de consulta.} &
Modelar un proceso de intercambio de información basado en servicios web en el
sistema. &
\multirow{3}{2.5cm}{Sistema de consulta de horarios durante el periodo de
clases.} \\
\cline{4-4}
& & & Implementar estándares de servicios web en el sistema. & \\
\cline{4-4}
& & & Diseñar e implementar persistencia de la información sobre el dispositivo
móvil. & \\
\hline
\end{tabular}
\caption{Ingeniería de proyecto}
\label{ingenieriadeproyecto}
\end{sidewaystable}

\section{Justificación}
Se ve un gran ahorro de tiempo, tanto para los estudiantes, que podrán
organizar mejor sus horarios, además de tenerlos a disposición en cualquier
momento.

En el aspecto social, promueve la comunicación y fomenta la comunión entre
personas con distintas restricciones de tiempo, haciendo que unos puedan conocer
y decidir que caminos pueden seguir, y a otros mostrando las ventajas y/o
desventajas que pueden encontrar en el camino a sus objetivos.

\section{Innovación tecnológica}
Se plantea utilizar la tecnología provista por node.js, para desarrollar en el
lenguaje de programación Javascript, tanto en su backend como en su frontend,
además de hacer uso del framework ionic para la construcción de la versión móvil
del sistema.

Así también se pretende utilizar toda la estructura provista por
REST\footnote{Representational State Transfer (REST): Técnica de arquitectura
software para sistemas hipermedia distribuidos como la World Wide Web.} para la
implementación de servicios, tanto para provisión como consumo de recursos web.

Además se facilitara la implementación del sistema sobre cualquier
infraestructura con la herramienta docker para virtualización ligera.

\section{Alcance}
El desarrollo de este sistema considera la construcción de los servicios,
notificaciones de eventos por el lado del servidor, es decir el backend del
sistema; la construcción de la aplicación web, y la aplicación móvil híbrida
como parte del frontend del sistema.

\end{document}

